\section{February}

\subsection{How do Tokamaks work and also fusion?}

\subsection{The types of drifts.}
The $E \cross B$ and polarization drift are described as, 
\begin{align}
	\bold{v}_{E \cross B} &= -\frac{1}{B} \nabla \phi \cross \hat{z}, \\
	\bold{v}_{P} &= - \frac{1}{\omega_{ci} B} \frac{d}{dt} \nabla \phi. 
\end{align}
Where $d/dt$ is defined as, 
\begin{align}
	\frac{d}{dt} = \frac{\partial}{\partial t} - \frac{1}{B} (\nabla \phi \cross \hat{z}) \cdot \nabla + \dots
\end{align}
The $\dots$ refer to higher-order corrections. 


\subsection{Hasegawa-Mima derivation and origins.}
Drift waves and their nonlinear interactions are one of the most fundamental elementary processes in magnetized inhomogenous plasmas. The simplest model equation that includes a fundamental nonlinear process is known as the Charney-Hasegara-Mima equation. The advective nonlinearity (Lagrange nonlinearity) associated with $E \cross B$ motion plays a fundamental role in drift wave dynamics. This nonlinearity appears in the fluid description as well as in the \textcolor{blue}{Vlasov} description of plasmas.

The simplest model equation is constructed for the inhomogenous (Opposite of homogenous, which means it is uniform without irregularities) slab plasma, which is magnetized by a strong magnetic field in the $z$-direction. There is also a density gradient in the $x$-direction. Plasma temperature is constant, and temperature pertubations are not considered. Ion temperature is assumed to be much smaller than that of electrons. The pertubations is manly propagating in the ($x,y$) plane, and has a small wave number in the direction of the magnetic field $k_z \ll k_{\perp}$. The small but finite $k_z$ is essential, so that the drift wave turulence is a quasi-two-dimensional turbulence. The electrostatic pertubation $\tilde{\phi}$ is considered. Under these specifications, the dynamical equation of plasmas is investigated and the nonlinear equation is deduced. 

First, the electron response is considered. The thermal velocity is taken to be much faster than the phase velocity of waves, $v_{Te} \gg \omega/\abs{k_z}$, so that the pressure balance of electrons along the magnetic field line provides the Boltzmann response of electrons as, 
\begin{align}
	\frac{\tilde{n_e}}{n_0} = \frac{e \tilde{\phi}}{T_e},
\end{align}
where $n_0$ is the unpertrubed density and $T_e$ is the electron temperature. The ion dynamics is studied by employing the continuity equation, 
\begin{align}
	\frac{\partial }{\partial t} n_i + \nabla \cdot (n_i v_{\perp}) = 0,
\end{align}
and the equation of motion, 
\begin{align}
	m_i \frac{d}{d t} v_{\perp} = e (- \nabla \phi + v_{\perp} \cross B).
\end{align}
Ions are \textcolor{purple}{immobile} in the direction of the magnetic field line. Time scales are assumed to be much longer than the period of the ion cyclotron motion, and the equation of motion is solved by expansion with respect to $w_{ci}^{-1} d/dt$, where $w_{ci} = eB/m_i$ is an ion cyclotron frequency. 

To be continued \cite{diamond_itoh_itoh_2010}...

\subsection{Work thus far with pseudo-spectral code.}

\subsection{What is multiple scale perturbation?}

\subsection{What is Vlasov description of plasmas?}

\subsection{What is an inverse cascade?}

\subsection{What is multiple-scale analysis?}

\subsection{What is Landau damping?}

\subsection{What is Shear Stress?}
According to sources, the origin of turbulence comes from shear stress. Turbulent fluid motion is described as an irregular condition of flow in which the various quantities show a random variation with time and space coordinates, so that statistically distinct average values can be discerned (key feature: random fluctuations, which means non-periodic, or also secondary motion). Another part mentions that turbulence originates as an instability of laminar flows if Re becomes too large. Laminar flow (or streamline flow) occurs when a fluid flows in parallel layers, with no disruption between the layers. At low velocities, the fluid tends to flow without lateral mixing, and adjacent layers slide past one another like playing cards. Reynolds number (Re) is described as the ratio of inertial forces to viscous forces within a fluid which is subjected to relative internal movement due to different fluid velocities. 

\shear

With respect to laminar and turbulent flow regimes:
\begin{itemize}
	\item laminar flow occurs at low Reynolds numbers, where viscous forces are dominant, and is characterized by smooth, constant fluid motion;
	\item turbulent flow occurs at high Reynolds numbers and is dominated by inertial forces, which tend to produce chaotic eddies, vortices and other flow instabilities.
\end{itemize} 

The Reynolds number is defined as, 
\begin{align}
	\text{Re} = \frac{u L}{\nu}.
\end{align}

Final note, when shear stress is applied to the surface of the fluid, the fluid will continously deform, i.e. it will set up some kind of flow pattern inside the. Essentially shear stress is the opposite of the normal force that we are taught in undergraduate first year physics. Shear stress is a force that acts parallel to the body. 
